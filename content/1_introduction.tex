\chapter{Introduction}

In a world of unstructured data, \gls{nosql} Databases are of never-ending
popularity.  While \acrshort{sql} databases are still the most popular, two out
of the ten most used databases already are \gls{nosql}
\parencite{dbenginesRanking}.  The trends show that the interest in \gls{nosql}
increases a lot faster than in traditional databases \parencite[section
\texttt{ranking\_categories}]{dbenginesRanking}. That shows that it is
necessary to consider \gls{nosql} databases as alternatives for relational
databases and to discuss their advantages as well as their disadvantages.

The hype around \gls{nosql} Databases in today's landscape can not be
disregarded.  The popularity for \gls{nosql} databases has been steadily
increasing and gaining fans from all over the database community.

But the term \gls{nosql} was first used in 1998 already, introduced by Carlo
Strozzi. While the term can mean either 'not only SQL' or 'no SQL', it is
uniform in the meaning of not being a relational database system. Soon after,
in the beginning of the next century, \gls{nosql} really started gaining
momentum and first implementations of different technologies started coming up.
One of the first \gls{nosql} technologies to come up was the document oriented
database implementation Metakit, soon followed by the first ever graph database
Neo4j in the year 2000. After that, many more followed and \gls{nosql}
databases have become a central point to the database landscape today. 

While it is clear that \gls{nosql} databases will not replace all other
\glspl{dbms} in existence today, they are especially popular in a few distinct
scenarios: For instance, when prototyping a new application while the data
format is still undecided, \gls{nosql} databases provide developers with
flexibility unseen in usual relational \glspl{dbms}. In addition to that, the
fast response and processing times can drastically increase an application's
performance, if used correctly. Also, the ability to store data which does not
adhere to a constant schema is significant with ever-changing and
uncontrollable data sources.

With the increasing amount of databases offered, choosing the right one can be
overwhelming. Not only does one need to decide for the right data structure,
but also find an implementation that fits the needs of your application.

Due to this growing importance of NoSQL databases in today's IT-landscape and
the abundance of different technologies implementing NoSQL, it is of great
importance to provide some guidance for this topic. For this reason this book
delivers not only an overview over some of the most popular NoSQL databases but
also analyses their features in the context of the CAP-theorem developed by
Brewer.

This eBook gives an overview of different types of \gls{nosql} technologies
that are relevant today. First, it will talk about Key-Value databases
including a quick general introduction and the selected implementations
Hazelcast, Redis and Riak. After that, Column Wide Oriented databases are
introduced, including the basics of Cassandra. Chapter 4 focuses on Documented
Oriented databases like Couchbase and Rethink DB. Lastly, the principals of
Graph databases are described with Neo4j as an example implementation. 


